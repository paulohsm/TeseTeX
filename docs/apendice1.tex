%%%%%%%%%%%%%%%%%%%%%%%%%%%%%%%%%%%%%%%%%%%%%%%%%%%%%%
%Apêndice A
\hypertarget{estilo:apendice1}{} %% uso para este Guia
%Este apêndice foi criado apenas para indicar como construir um apêndice no estilo, não existia no original da tese.
%%%%%%%%%%%%%%%%%%%%%%%%%%%%%%%%%%%%%%%%%%%%%%%%%%%%%%
\renewcommand{\thechapter}{}%
\chapter{APÊNDICE A - DESCRIÇÃO DO MODELO ACOPLADO}	% trocar A por B na próxima apêndice e etc
\label{apendiceA}	% trocar A por B na próxima apêndice e etc
\renewcommand{\thechapter}{A}%		% trocar A por B na próxima apêndice e etc

O modelo empregado nas simulações do clima é o BESM 2.4 ...


\section{Componente atmosférica}


\subsection{Representação das nuvens} 


\section{Componente oceânica}

